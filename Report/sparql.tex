\chapter{Query SPARQL}
%\addcontentsline{toc}{chapter}{Stato dell'arte}
SPARQL (acronimo ricorsivo di SPARQL Protocol And RDF Query Language) è un linguaggio di interrogazione standardizzato e interoperabile per i dati rappresentati tramite RDF. Soddisfa o supera l’SQL nelle sue capacità e potenza, pur mantenendo gran parte della sintassi familiare ed è uno degli elementi chiave delle tecnologie legate al paradigma noto come web semantico consentendo l'estrazione di informazioni dalle basi di conoscenza distribuite sul web.
\\\par
Sono state eseguite delle query sull'ontologia prodotta per poter effettuare un ulteriore controllo e valutare se i risultati ottenuti sono coerenti con quelli attesi. Di seguito vengono mostrate alcune delle query eseguite.
\\\par
\begin{figure}[h]
	\centering
	\includegraphics[width=12cm]{Query1.JPG}
	\caption{Lista di tutti i modelli di gomme}
\end{figure}
\clearpage
\begin{figure}[h]
	\centering
	\includegraphics[width=12cm]{Query2.JPG}
	\caption{Lista di tutti i partecipanti con relativi veicoli e gomme montate dai veicoli}
\end{figure}
\clearpage
\begin{figure}[h]
	\centering
	\includegraphics[width=12cm]{Query3.JPG}
	\caption{Lista dei partecipanti, round e relative ore di inizio e fine turno, ai TrackDay con prezzo maggiore di 300 euro}
\end{figure}
\clearpage
