\chapter{Conclusione}
La realizzazione di questo progetto ha permesso di approfondire e mettere in pratica i concetti appresi a lezione e rappresenta un buon esempio delle potenzialità offerte dal web semantico, il cui scopo principale ricordiamo essere quello di collegare il più possibile i dati e renderli interpretabili dalle macchine. È stato approfondito inoltre l'utilizzo del software open source Protégé, punto di riferimento per lo sviluppo di ontologie.
\\
Questa ontologia è anche un buon punto di partenza per poter realizzare nuove ontologie. Di seguito sono elencati i possibili miglioramenti al progetto:
\begin{itemize}
\item Inserimento di classi per modellare caratteristiche tecniche del tracciato (Tipologia asfalto, dislivello assoluto, anno di realizzazione, … );
\item Inserimento di classi per modellare i tempi sul giro;
\item Inserimento di classi per modellare tutti i dati atmosferici collegati al tracciato e che influiscono sulle prestazioni (Temperatura aria, temperatura asfalto, condizioni meteorologiche, direzione e intensità vento, … );
\item Inserimento di classi per migliorare la rappresentazione dei veicoli (motore, telaio, freni, peso, tipologie di gomme, … );
\item Inserimento di classi per modellare il setup dei veicoli (camber, convergenza, ripartizione frenante, ripartizioni peso, tipologia rapporti, pressione gomme, … ).
\end{itemize}




