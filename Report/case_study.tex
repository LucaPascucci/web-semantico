\chapter{Track Day Event Ontology}
Lo scopo di questo progetto è di approfondire ed utilizzare le tecnologie del Web Semantico per creare un'ontologia. Verrà descritto passo per passo come è stata realizzata, quali strumenti sono stati utilizzati e quali tecniche impiegate. 
\\\par
L'ontologia che si intende modellare riguarda gli eventi di Track Day che rappresenta per molti appassionati di motori la possibilità di girare in pista con la propria auto a prezzi solitamente contenuti. In particolare si vuole avere la possibilità di definire una pista con le sue caratteristiche principali (come ad esempio luogo, lunghezza, curve a destra o sinistra) dove si terrà l’evento, l’organizzatore responsabile dell’evento che può essere una persona come piuttosto un’organizzazione. Nell’evento vengono organizzati differenti turni per dare a tutti i partecipanti la possibilità di girare in pista. Esempi di triple di informazioni possono essere “L’evento X ha come partecipante la persona Y”, oppure “La persona Y ha come veicolo il mezzo Z” e così via.
Una prima modellazione è stata effettuata con RDF/RDF Schema, che ci permette di esporre la sintassi per definire schemi e vocabolari per i metadati, oltre a poter definire una serie di triple come quelle mostrate precedentemente, tramite espressioni nella forma soggetto-predicato-oggetto.
\\\par
Tramite RDF/RDFS è possibile definire per esempio:
\begin{itemize}
\item rdfs:class: Definisce gruppi di individui che condividono alcune proprietà comuni. Ad esempio Luca e Filippo possono essere entrambi istanze della classe \textit{Participant}, come si vedrà in seguito;
\item rdfs:subclassOf: Fornisce la possibilità di organizzare le classi in gerarchie di specializzazione. Per esempio, la classe \textit{SemiSlickTyres} è sottoclasse della classe \textit{Tyres};
\item rdf:property: Rappresenta la parte di predicato della tripla soggetto-predicato-oggetto. Per esempio, le classi \textit{Participant} e \textit{Vehicle} sono collegate dalla proprietà \textit{hasVehicle} (Participant hasVehicle Vehicle);
\item rdf:domain: Definisce la classe che fa da soggetto per la proprietà;
\item rdf:range: Definisce la classe che fa da oggetto per la proprietà.
\end{itemize}

